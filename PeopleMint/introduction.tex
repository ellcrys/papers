\section{Introduction}
\vspace{7mm}

In this section, I will discuss current coin generation and reward models used on existing
public blockchains based on their consensus mechanism and why there is a need for alternatives that will offer a more egalitarian, accessible and inclusive coin distribution model. 
\subsection{Proof of Work (PoW)}
\paragraph{} {
	In most blockchain systems, especially those based on Proof-of-Work (PoW) like Bitcoin\cite{bitcoin}, the distribution of the native coin happens via a repeating process involving the computation of cryptographic hashes in search of a given value that meets specific criteria. The participants who find these unique values are allocated newly created coins. This process is popularly known as "Mining", and it requires specialised hardware to keep the miners competitive. Many in the cryptocurrency industry widely criticise it because of the belief that it is an unworthy contributor to energy waste and ultimately damaging for the environment. Furthermore, the introduction of ASIC miners reduce miner inclusivity by centralising and delegating coin generation operation to corporations. As a result, mining increasingly became out of reach of most people. 
	
\subsection{Proof of Stake (PoS)}
	\paragraph{}{In Proof-of-Stake systems, the need to repeatedly compute random values that meet a set of criteria is not required. Instead, network participants who are interested in creating blocks stake some money (usually in the native coin) which is used as a security deposit to discourage bad behaviour. Upon the creation of a new block, the block creator is rewarded with some coins. This coin generation method outrightly eliminates users who are not able to afford the minimum staking amount. This will lead to a system that favours only the rich.}
	Some PoS systems like Tezos\cite{liquid-pos}},  participants who are unable to meet the staking requirement have the ability to use their coin to vote delegates who would create blocks on their behalf and with the possibility of sharing the block rewards with them.

\subsection{Hybrid PoW/PoS}
	\paragraph{}{
		Hybrid consensus system that successfully integrate features of both Proof-of-Work and Proof-of-Stake behaviours increase the number of participants who are eligible to benefit from the network coin creation process. An exciting example is Decred network, which allows PoW miners create blocks while PoS miners endorse those blocks. In this system, PoS miners stake money by purchasing a ticket which may be randomly selected from a pool to take part in the block endorsement process. Network participants who are unable to compete successfully on the PoW side may choose to purchase tickets and potentially be called upon to validate blocks and receive new coins as a reward. Hybrid consensus system offers a more inclusive and equatable coin distribution mechanism than Proof-of-Work system. 
	}
	
	










 